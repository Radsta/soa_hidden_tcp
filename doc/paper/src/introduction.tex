% vim: set tw=78 sts=2 sw=2 ts=8 aw et ai:

Steganography is the science of encoding messages in such a way that only the
sender and receiver can read them. It’s different from cryptography in the
sense that you don’t actually encode the message with a key or an algorithm,
instead hiding the message in things like images (each letter in a pixel, far
from each other enough so it doesn’t raise suspicion), audio files (a certain
tone at a certain interval can signify a letter) and even plain text
(repetition of certain letters can translate into a hidden message).

Although steganography has been used as early as Ancient Greece, the constant
development of technology has spawned a new field of research, namely digital
steganography. This focuses more on applying steganography in sending messages
over a network, either for increased security (plain cryptography can be
cracked if the attacker figures out what type of encryption you are using),
or for sending messages in such a way the network administrator is unaware
of them (for trivial tasks that would otherwise cost the user money, e.g. in a
cloud environment).

The purpose of this paper is to offer a solution for hiding messages in the
TCP/IP header. We aim to offer both a decent amount of space to store the
message and an encoding that will minimize the size of the message.
