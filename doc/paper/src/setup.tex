% vim: set tw=78 sts=2 sw=2 ts=8 aw et ai:
We have performed three rounds of experiments in order to test the efficiency and
robustness of our system. Our testbed consists of two Linux boxes running Debian 7.3
"wheezy" with Linux kernel 3.12-amd64.

In the first round of experiments, we ran the server and the client on the same
station using the loopback interface to send packets. In this way we have removed
the danger of middlebox interference, focusing only on the functionality of our
implemented modules. Tests consisted of sending several commands from the client
and receiving them on the server, followed by relaying back the output to the client.
We have covered various cases such as short commands, long commands that needed to
be split up into several packets, invalid commands that needed to be caught by
our parser, commands that violated the privilege level, commands read from standard
input or from a script given as argument to the client.

The second round of experiments targeted middlebox interference in a virtual
environment. For this we use Qemu with Cisco ASA 9.0(1) inside our Debian system.
The version of Qemu we use is a modified 14.0. This setup comprises three Qemu VMs:
a client, a server and an ASA firewall running Cisco IOS version 15.1(1)SY between
them. Packets could be sent between the client and the server using virtual
interfaces and could be monitored from inside the firewall. We experimented using
the same set of tests from the first round.

The final set of tests concerned itself with middlebox interference in the real
world. This time we sent packets between two public IP addresses with three real
routers inbetween. The drawback of this setup is the lack of control over the
middleboxes and the inability to determine the exact cause of eventual failures.
