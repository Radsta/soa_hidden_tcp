% vim: set tw=78 sts=2 sw=2 ts=8 aw et ai:

Evolution of today's computer science demands increasingly new and intelligent
approaches. Steganography is a field with a very long history. Starting from
the antique wax tablets, continuing with invisible ink, steganography has also
evolved, so that nowadays we can speak of digital steganography.

Our project successfully adds up to this research direction. We have managed to
implement a novel approach, we have a working steganographic communication and,
as far as we know, the biggest amount of hidden data transmitted inside a
single segment. We have used the TCP header, more precisely the "Options"
field, sending a total of 37 bytes per TCP segment.

As future work, we would like to follow a few directions. First, we would like to
see how the application behaves in a controlled, larger system, so that we may solve
the problems that occured when we tried to send packets between two public IPs. If
the problem is related to devices not supporting the extra options field, we might 
consider sending packets to intermediate stations, on paths that we know are safe
and will not cause those packets to be dropped. Second, as you have seen, using
Huffman encoding took us a large number of bytes to assure the necessary metadata
(codebook, message length, etc). Future studies will include finding and
implementing better compression algorithms to minimize the size of the
communication. Third, cryptography will be a future thing to be added for
improved security, in order to add a new level of hiding to our steganographic data.
