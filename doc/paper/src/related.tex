% vim: set tw=78 sts=2 sw=2 ts=8 aw et ai:

One attempt to use steganography for hiding messages in TCP/IP headers was
presented in 2010 at ICACTE by a Dhobale et al\cite{stegano-hiding-data}. They had 2
scenarios in mind.

On the one hand, using the Flags section of the IP header, particularly the DF flag, to send
1 or 0 to the receiver, thus prompting an action. This scenario only works if
the MF flag is set on 0, which makes the network receivers along the way to
ignore the DF flag. Although this method works, it's not suitable for longer
messages because of the number of packets you would have to send over the
network.

On the other hand, using the IP identification field to send a message encrypted using chaotic
mixing. This method doesn't work if along the way a checker wants the field to
contain certain information and it also splits messages into pieces, even
smaller ones like "ABC", which can cause suspicion.

\begin{figure}
  \centering
  \includegraphics[width=0.5\textwidth]{img/IPHeader}
  \caption{IP header structure}
  \label{fig:related}
\end{figure}

SCONeP\cite{ciobanu2011sconep} is another example of state of the art in the
field of steganography. SCONeP stands for Steganography and Cryptography over
Network Protocols and offers an implementation of a protocol which provides
the protection of hidden data by encrypting it.

Their application is able to send, receive and detect encrypted messages
encapsulated inside TCP, IP, ICMP and UDP headers. They also use compression
via Huffman trees for the messages. RSA plus Vigenere or Triple DES are used
for encryption.

Regarding the TCP header part of the project, SCONeP uses the two 32-bits
fields from the header, sequence number and acknowledgement number. Relating
to them, we are using the options field in order to achieve a better length
and a broader freedom for the message sent.
