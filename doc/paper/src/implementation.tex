% vim: set tw=78 sts=2 sw=2 ts=8 aw et ai:

Proper implementation is proper.

Another module of our project is represented by the Encoding module. As we
have previously mentioned, the TCP header options field offers us 37 useful
bytes. Although this is a greater amount than other protocols offer, it is
still a limited amount, so we have to use it properly.

The first thing to do is minimize the size of the input data (the useful
data encapsulated inside the TCP header), which leads to the idea of compressing
it. Our communication must be reliable and entirely exact, pointing to an
approach based on a lossless data compression algorithm. One of the most
suitable lossless algorithms is Huffman coding.

Huffman coding\cite{huffman1952method} is an entropy encoding algorithm using
a variable-length code table for encoding source symbols (characters in the
current situation), where the table is generated based on the probabilities of
occurence for each of the source symbols. In the end, what you get is a binary
tree of nodes which provides you unique combinations of bits for all the
symbols.

Practically speaking, the compression technique works in the following way:
first, all the nodes are leaf nodes, which contain the symbol itself and the
frequency of occurence; then the nodes are taken two by two starting with the
smallest frequencies, joining them by a common parent containing the sum of
their frequencies; the process is repeated until you have a single root node;
the edges are given bit values (0 the left branches, 1 the right branches);
the encoding for each letter is calculated by following the path of bits from
the root to the corresponding leaf.
